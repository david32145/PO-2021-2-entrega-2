\documentclass[]{article}
\usepackage[brazil]{babel}
\usepackage[utf8]{inputenc}
\usepackage{graphicx}
\usepackage{xcolor}
\definecolor{green}{rgb}{0,0.5,0}
\usepackage{amsmath}
\usepackage{amsfonts}
\usepackage{amsthm}
\usepackage{listings}
\usepackage{tikz}
\usepackage{hyperref}
\usepackage{verbatim}
\usetikzlibrary{arrows,shapes}

\numberwithin{equation}{section}

%opening
\title{Pesquisa Operacional}
\author{Francisco David Nascimento Sousa}

\begin{document}

\section{Questão 84}

O problema do 3-SAT consiste em uma conjução de cláusulas, ou seja, 3-SAT= $C_1 \wedge C_2 \wedge C_3 \wedge ... \wedge C_n$, e cada cláusula possui 3 literais, ou seja, $C_i = (l1 \vee l2 \vee l3)$, além disso cada literal pode ter o valor
de $x_i$ ou $\overline{x_i}$, onde $x_i$ é uma das m váriveis lógicas do problema.

O primeiro passo da "redução" é mapear cada variável lógica em uma variável inteira, usando 1 e 0 para representar verdadeiro e falso respectivamente. Ou seja, no modelo haverá uma variável inteira $z_i$ para cada variável lógica, com a seguinte restrição:

\begin{align}
     \forall{z_i}, 0 \leq z_i \leq 1
\end{align}

Com isso, é garantido que $z_i$ terá os valores de 0(falso) ou 1(verdadeiro).

Outro ponto importante é que cada cláusula precisa ser verdadeira, no problema do modelo vamos transformar a disjunção em operações de soma, de modo que:

\begin{align}
   C_i = (x_1 \vee x_2 \vee x_3) = true \Leftrightarrow (z_1 + z_2 + z_3) \geq 1
\end{align}

Ou seja, para cada cláusula haverá essa restrição:

\begin{align}
     z_1 + z_2 + z_3 \geq 1 \\
     z_4 + z_5 + z_6 \geq 1 \\
     ... \\
     z_{m-2} + z_{m-1} + z_m \geq 1
\end{align}

Além disso, existe o caso em que a variável lógica está negada, ou seja, $\overline{x}$, nesse caso queremo que quando x for verdadeiro $\overline{x}$ seja falso e vice-versa, para isso precisamo trocar cada ocorrência de $z_i$ em que $x_i$ era $\overline{x_i}$ por $1 - z_i$.

Portanto, temos o seguinte modelo:

\begin{align}
&\quad f = (z_1 + z_2 + z_3) + (z_4 + z_5 + z_6) + ... + (z_{m-2} + z_{m-1} + z_m) \\
\text{s.t}  &\quad  (z_1 + z_2 + z_3) \geq 1 \\
            &\quad  (z_4 + z_5 + z_6) \geq 1 \\
            &\quad  ... \\
            &\quad  (z_{m-2} + z_{m-1} + z_m) \geq 1 \\
            &\quad \forall{z_i}, 0 \leq z_i \leq 1
\end{align}

Claramente a ordem das variáveis não deve seguir essa ordem, mas foi mais para ilustrar. O importante é garantir que elas vão ser inteiras ou 0 ou 1, e que a cláusula é verdadeira com a restrição da soma ser pelo menos 1.

Tal problema percetence a classe de problema NP Difíceis, pois foi possível "reduzir" o 3-SAT para esse problema de programação linear inteira em operações que levam tempo polinomial. Então se fosse possível resolver esse problema em tempo polinomial, o 3-SAT também seria, o que certamente não é possível.

\section{Questão 85}

Como $z$ está bem definido, então existe um $x_1$, onde $c^Tx_1 = z$, e como:

\begin{align}
     Ax_1 \leq b, \implies Ax_1 - b \leq 0
\end{align}

Como $u \geq 0$ e $Ax_1 - b \leq 0$,então $u^T(Ax_1 - b) \leq 0$, então
$z = c^Tx_1 \leq c^Tx_1 - u^T(Ax_1 - b) = z(u)$.

Logo, $z \leq z(u)$.

\section{Questão 86}

Temos os seguintes modelos:

\begin{align}
z =   \max        &\quad  c^Tx       \\
      \text{s.t}  &\quad  Ax \leq b  \\
                  &\quad  x \geq 0
\end{align}

\begin{align}
z_{UB} =  \min        &\quad  u^Tb        \label{startDLP} \\
          \text{s.t}  &\quad  A^Tu \geq c                  \\
                      &\quad  u \geq 0    \label{endDLP}
\end{align}

Como $z$ está bem definido, então existe $x_1$, ondex $z = c^Tx_1$, e:

\begin{align}
     Ax_1 \leq b
\end{align}

Vamos multiplicar (3.7) por $u^T$, teremos:

\begin{align}
     u^TAx_1 \leq u^Tb
\end{align}

Como $z_{UB}$ também está bem definido, podemos dizer que $z_{UB} = u^Tb$, e:

\begin{align}
    A^Tu \geq c
\end{align}

Podemos transpor ambos os lados (3.9):

\begin{align}
    (A^Tu)^T \geq (cx_1)^T
\end{align}

Podemos multiplicar por $x_1$ (3.9):

\begin{align}
    (A^Tu)^Tx_1 \geq (cx_1)^Tx_1 \implies  u^TAx_1 \geq c^Tx_1
\end{align}

Assim temos:

\begin{align}
    z = c^Tx_1 \leq u^TAx_1 \leq u^Tb = z_{UB}
\end{align}

Logo, $z \leq z_{UB}$.

\section{Questão 87}

\begin{align}
z_{UB} =    &\quad  u^Tu       \\
\text{s.t}  &\quad  A^Tu \geq c  \\
            &\quad  u \geq 0
\end{align}

Vamos chamar de $z$ a relaxação, dada por:

\begin{align}
z = &\quad  u^Tu - x(A^Tu - c)       \\
\text{s.t}  &\quad  u \geq 0
\end{align}

Vamos fazer o mesmo processo que o outro, e vamos chamar de $z_{LB}$

\begin{align}
z_{LB} = \max z(x)       \\
\text{s.t}  &\quad  u \geq 0
\end{align}


\begin{align}
z_{LB} = \max_{x \geq 0} \min_{u \geq 0} \quad u^Tb - x(A^Tu - c)       \\
z_{LB} = \max_{x \geq 0} \min_{u \geq 0} \quad u^Tb - xA^Tu + xc       \\
\end{align}

Vamos isolar os termos independentes:

\begin{align}
z_{LB} =  \max_{x \geq 0} &\quad  xc +
          \left( \min_{u \geq 0} \quad u^Tb - xA^Tu \right)  \\
z_{LB} =  \max_{x \geq 0} &\quad  xc +
          \left( \min_{u \geq 0} \quad (b^T - Ax)u \right)  \\
z_{LB} =  \max_{x \geq 0} &\quad  xc +
          \left( \min_{u \geq 0} \quad (b - Ax)^Tu \right) \\
z_{LB} =  \max_{x \geq 0} &\quad  c^Tx +
          \left( \min_{u \geq 0} \quad (b - Ax)^Tu \right) 
\end{align}

Basta uma restrição de que $Ax \leq b$ e notar que $\min$ vai ser igual a 0, logo temos que:

\begin{align}
z_{LB} =    &\quad  c^Tx       \\
\text{s.t}  &\quad  Ax \leq b  \\
            &\quad  x \geq 0
\end{align}

Que basicamente é a equação incial, então o dual do dual é ele mesmo.

\end{document}
